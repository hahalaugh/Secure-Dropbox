\chapter{Introduction}
Cloud computing is a model for enabling ubiquitous, convenient, on-demand network access to a shared pool of configurable computing resources that can be rapidly provisioned and released with minimal management effort or service provider interaction \cite{Mell2011}. It is by definition composed of three service models:

\begin{itemize}
\item
Cloud Software as a Service (SaaS) provides computation capacity by running software on a cloud infrastructure. These applications are remotely accessible to users. The users are not supposed to manage the cloud infrastructure hardware but only utilize the services. Different cloud consumers' applications are organized in a single logical environment in the SaaS cloud to achieve economies of scale and optimization regarding issues like speed, security, availability, disaster recovery, and maintenance\cite{Mell2011}\cite{Dillon2010}. Those cloud storage services which this research mainly aims at, Dropbox and Google Drive for instance, are representative examples of SaaS.
\item
Cloud Platform as a Service (PaaS) is a development framework and service hosting platform which allows users to develop cloud services by providing runtime environment, R\&D toolkits, SaaS application APIs, storage and middleware/OS support. An example instance of PaaS is the Google Application Engine.
\item
Cloud Infrastructure as a Service (IaaS) refers to on-demand provisioning of infrastructural resources on the cloud, usually in terms of VMs\cite{Zhang2010}. The capability provided to the consumer is about provision processing, storage management, network configurations and other fundamental computing resources. The customers are able to deploy and run any software in the cloud\cite{Mell2011}\cite{Zhang2010}. Amazon EC2 is a representational IaaS cloud platform.
\end{itemize}
Cloud storage refers to the service on the cloud which is delivered as remote storage infrastructure. It can be accessed from any terminal connected to the Internet. It also offers additional data-related functionalities regarding to data management and maintenance\cite{Cachin2009}. Any operations in which the core tasks that a cloud computing system processes are relevant to big data storage and maintenance, it could be defined as a cloud storage system. Most hardware configuration and software deployment inside are optimized aiming for providing data storage service. Just as it is a service that provides interfaces in terms of data manipulations rather than utilizing certain physical storage devices directly, cloud storage is by nature an application of SaaS.

For personal users, cloud storage is an extension of local storage with good cost performance. Also it could be an ideal replacement of portable storage devices like flash drive or portable hard drive by which stable data storage is not guaranteed. Moreover, it is also a more reliable local disk backup given its essence of good cost efficiency, remote accessibility and reliability. Besides, from the enterprise-level users’ perspective, cloud storage reduces the investment they should have made in terms of design, management and maintenance of device clusters to build a systematic big data storage service. Lastly, the data sharing is another thing that boosts up data usage efficiency via facilitating the real time data exchange and decreasing the workload about data collection especially relating to frequently reused data resources.

As a highly developed commercial-level cloud storage product and innovator in this field, Dropbox is becoming the industry leader with a market share of 14.14\% in 2011\cite{OPSWAT2011}. Google Drive, another popular cloud storage service which is built based on application framework by Google and well integrated with other Google services (e.g. Gmail as a major way of acting user or system activity notification) is also increasing dramatically. It is routine for cloud storage service vendors to provide multiple ways of getting access to the service from different platforms for ubiquitous usage. For example, desktop client, web application and portable device application.

\section{Motivation}

Alongside the high speed development of the cloud storage industry, security anxiety in the age of the cloud is becoming a significant issue. There are frequent reports of web servers being attacked: either because data residing on servers is of interest to hackers, or because the hackers want to leave their mark of exploiting. While robustness and availability are main concerns of security, confidentiality is equally important\cite{Beaver2003}. Data storage security problem is an important aspect of Quality of Service(QoS)\cite{Kumar2010}.The security features of cloud storage technology are always compared with traditional ways like local disk or RAID. For personal users, the reasons for using the cloud storage service or not are simple and straight forward. For instance, would people put their security sensitive data like bank account information on the cloud? Also, even if service carriers claim that they deploy secure encryption scheme throughout, is it definitely sure that the employees of service carriers will abide strictly by the terms of service and never try to access users’ data unconsciously by technical approaches which could be easily performed internally? In addition, some personal users with computer expertise could have concerns about whether the continuous on service cloud storage server might represent an attractive attack target and therefore be riskier than local storage. Lastly, personal users, particularly those who use cloud storage service for free, have concerns about who will bear responsibility for any loss of personal data whilst it is stored on the cloud. Such problems relating to security confidence restrict users to only storing unimportant data on the cloud as redundancy and utilize the feature of portability. Even if the cloud storage service provider could be Google or Dropbox, it is paradoxical that people only achieve redundancy for important data because the security concerns hinder it.

For enterprise users, except for the same concerns that might be taken into consideration as a personal user, there are more severe problems to be dealt with. According to the survey hosted by Intel in May 2012\cite{Intel2012}, the concerns of IT professionals regarding public cloud (opposite to the private cloud and traditional IT storage solutions like local disk or RAID. The storage service provided by Google Drive and Dropbox discussed in the paper are typical instances of public cloud) were: Firstly, the lack of measurement of security capabilities of cloud storage service. Despite the security methods deployed which claim to be secure, the providers hide the implementation detail and the only transparent approach to users is simply identity based access control. It is not sufficient to gain their confidence towards the security of the storage system. 57\% of the survey participants thought it was the most significant problem concerned when using such service among all the concerns. 55\% of the participants considered a lack of control over data as another major concern because of the invisible abstracted resources and shared storage infrastructure in the cloud. Only 36\% of the participants thought the cloud storage service lack of transparency/ability to conduct audits which indicates that the continuously improving of comprehensive query interfaces about storage service itself makes it acceptable by gaining the level of transparency. Compliance with regulatory mandates is another key problem concerned by an average of 78\% participants. Due to the opaque security and operability of cloud storage, the security of storing legislation sensitive data might not be fully measured and leads to compliance problems. Such laws vary from country to country. One of the advised solutions is making full use of cloud storage for enterprise data is to classify the data with different priorities and keep them in internal infrastructure and external storage services respectively.

The goal of the project is to investigate the feasibility of gaining users’ confidence towards the security of cloud storage service by utilization of a hybrid encryption mechanism in the local host. Also to evaluate the advantages of information sharing based on asymmetric cryptology over the symmetric cryptology. To achieve this goal, it was decided to develop a desktop client which performs local file management and cryptology operations and a server, which functions as a key management service, user management system and a file sharing pool. The combination of the two components will implement the entire procedure as a prototype.

On completion of the project, the production will be demonstrated to information security experts and several testers who do not have Computer Science background will be invited. The evaluation will be mainly about performing user acceptance tests for assessing the availability and user experience of the software.

\section{This Report}

This report will contain a review of the state of the art technologies in the fields of cloud storage, cloud security approaches and other potential solutions those are relevant to this project. Some other popular approaches will also be discussed, together with justification for their inclusion.

Important design decisions made during the undertakings of this course project will then be explained in detail, along with the reasons that justify these decisions. There will be followed by a detailed outline of both design and implementation of key components of the software. A systematic analysis of the software prototype will be given based on not only the expertise of the IT security professionals but also comments and suggestions by routine Dropbox users. Assessment of this application will focus on the theoretical correctness, availability and user experience based on the feedback from beta users. Additionally, potential future works including implementing such an application on portable platform and evaluating the possibility of performing a file system level encryption will be discussed. Finally, the conclusion of the whole project will be presented.
